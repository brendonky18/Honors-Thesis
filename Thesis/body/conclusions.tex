



By now, the potential utility and pedagogical value of hands-on work is common knowledge among educators, and it is common practice to incorporate some form of such work into cybersecurity curricula. %%



There is a vast body of both educational and cybersecurity literature specifically focusing on these subjects. %%



Unfortunately, much of the extant research sees this fact as \emph{non plus ultra}; %%



they believe that there is nothing further beyond to be explored. %%



While is is clear that interactive work can tremendously enhance the learning experience and educational outcomes for students, what's not incontrovertible is the assumption that this is the end-all be-all in terms of the discussion to be had. %%



The use of hands-on work is not the edge of the world, it is merely the horizon. %%



This is in fact a highly active area of educational research, it simply seems that cybersecurity educators have not gotten the same message. 







Among the available cybersecurity research, there has been a large amount of energy invested into the design of different frameworks. %%



However, this work focuses on enhancing various features, such as flexibility, ease-of-use, extensibility, or modularity. %%



This work has seen the conclusion that hands-on work can be greatly beneficial, and taken it to mean that any form of hands-on work is good. %%



As this research has shown, that assumption is simply not true. 







Not all hands-on work is made equal, and there a wide number of considerations and combinations thereof that can seriously impact learning outcomes. %%



This research sought to examine two of those parameters: is there a discernable difference between content-based learning, and skills-based learning; %%



and is there a difference in outcomes across varying level guidance. %%



It then attempted to examine how the possible differences should inform the future design of hands-on work, in order to maximize it's educational utility. %%



The research shows that there are strong indications to confirm both are true ---%%



 especially so for the latter. %%



There is a discernable difference between teaching specific factual knowledge, versus developing skills. %%



There is also a positive trend in favor of maximal guidance, suggesting that it is the approach most preferred by students, and that it is the approach that yields the most optimal learning outcomes. 







There are also many practical benefits to a maximally guided approach. %%



Namely, It ensures that students will spend a larger portion of their time actively engaged in learning, meaning that their time is being utilized most efficiently with respect to the amount of learning that is being achieved. %%



It also mitigates the possibility that students will find workarounds to the intended solution, which could detrimentally impact their learning outcomes. 











\section{Limitations}



    An unfortunate and obvious limitation of this research is the small sample size $(N=12)$. %%



However, even with such a small data set, it has produced some statistically significant results. 







            







\section{Future Work}







% further self selection bias, since students were informed that the activities involved hands-on work.

